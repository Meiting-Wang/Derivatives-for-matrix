%使用xelatex编译
%-------导言区----------------
\documentclass[UTF8,space=auto]{ctexart} %UTF8编码,设置其处理空格的方式为auto
\usepackage[a4paper]{geometry} %设置纸张为A4大小
\usepackage{hyperref} %可使用超链接(含链接的跳转和一些命令)
\usepackage{amsmath} %最常用的数学宏包
\usepackage{amssymb} %包含更多的数学符号
\usepackage{lipsum} %英语假文宏包
\usepackage{ifthen} %使用\ifthenelse命令
\usepackage[dvipsnames]{xcolor}
\usepackage{fancyhdr} %设置页眉页脚的宏包
\usepackage{bm}

%超链接设置(依赖于hyperref宏包)
\hypersetup{
	colorlinks=true,
	citecolor=magenta,%设置cite类命令超链接的颜色
	linkcolor=blue,%设置目录、脚注、ref等超链接的颜色
	urlcolor=violet,%设置网页超链接的颜色
}

%ctex宏包的设置
%\ctexset{
%	section = {
%		format+ = {\raggedright} %节标题左对齐
%	}
%}

%页眉页脚的设置 (依赖于fancyhdr宏包)
\pagestyle{fancy} %使用fancy风格
\fancyhf{} %清除所有的页眉页脚
\fancyhead[L]{\slshape\leftmark} %页眉RE、LO位置章标题
\fancyhead[R]{\slshape\rightmark} %页眉LE、RO位置节标题
\fancyfoot[C]{\thepage} %页脚居中页码
\renewcommand{\headrulewidth}{0.4pt} %重定义页眉线宽度
%\renewcommand{\footrulewidth}{0.4pt} %重定义页脚线宽度

%定义新命令或新运算符
\DeclareMathOperator{\diff}{d\!}
\newcommand{\var}{\mathrm{Var}}
\newcommand{\cov}{\mathrm{Cov}}

\allowdisplaybreaks[4] %允许多行公式跨页显示

%--------------------定理类环境重定义-------------------
%-设置计数器
\newcounter{mycntthm} %设置新的计数器
\newcounter{mycntexam}
\setcounter{mycntthm}{0} %设定计数器默认值
\setcounter{mycntexam}{0}

%-开关设置
\newcommand{\thmseriesnamestyle}{Chinese} % Chinese 表示设置定理类名称为中文,English 表示设置定理类名称为英文
\newcommand{\thmseriesnumbering}{true} % true 表示对定理类环境进行编号,false 表示不对定理类环境进行编号

\ifthenelse{\equal{\thmseriesnamestyle}{Chinese}}{%
	\newcommand{\dfnname}{定义}
	\newcommand{\lemmaname}{引理}
	\newcommand{\thmname}{定理}
	\newcommand{\coroname}{推论}
	\renewcommand{\proofname}{证明}
	\newcommand{\examname}{例}
}%
{%
	\newcommand{\dfnname}{Definition}
	\newcommand{\lemmaname}{Lemma}
	\newcommand{\thmname}{Theorem}
	\newcommand{\coroname}{Corollary}
	\renewcommand{\proofname}{Proof}
	\newcommand{\examname}{Example}
}

%-自定义定理类环境
\newenvironment{dfn}[1][]%定义环境
{%
	\par%
	\bfseries%
	\dfnname%
	\ifthenelse{\equal{\thmseriesnumbering}{true}}{\refstepcounter{mycntthm}~\themycntthm}{}%
	\ifthenelse{\equal{#1}{}}{~~}{~(#1)~~}%
	\mdseries\itshape\color{black}%
}%
{\par}

\newenvironment{lemma}[1][]%引理环境
{%
	\par%
	\bfseries%
	\lemmaname%
	\ifthenelse{\equal{\thmseriesnumbering}{true}}{\refstepcounter{mycntthm}~\themycntthm}{}%
	\ifthenelse{\equal{#1}{}}{~~}{~(#1)~~}%
	\mdseries\itshape\color{black}%
}%
{\par}

\newenvironment{thm}[1][]%定理环境
{%
	\par%
	\bfseries%
	\thmname%
	\ifthenelse{\equal{\thmseriesnumbering}{true}}{\refstepcounter{mycntthm}~\themycntthm}{}%
	\ifthenelse{\equal{#1}{}}{~~}{~(#1)~~}%
	\mdseries\itshape\color{black}%
}%
{\par}

\newenvironment{coro}[1][]%推论环境
{%
	\par%
	\bfseries%
	\coroname%
	\ifthenelse{\equal{\thmseriesnumbering}{true}}{\refstepcounter{mycntthm}~\themycntthm}{}%
	\ifthenelse{\equal{#1}{}}{~~}{~(#1)~~}%
	\mdseries\itshape\color{black}%
}%
{\par}

\newenvironment{proof}[1][]%证明环境
{%
	\par%
	\bfseries%
	\proofname%
	\ifthenelse{\equal{#1}{}}{~~}{~(#1)~~}%
	\normalfont\color{black}%
}%
{\vspace{-\baselineskip}\hfill$\blacksquare$\par\vspace{\baselineskip}}

\newenvironment{exam}[1][]%例环境
{%
	\par%
	\bfseries%
	\examname%
	\ifthenelse{\equal{\thmseriesnumbering}{true}}{\refstepcounter{mycntexam}~\themycntexam}{}%
	\ifthenelse{\equal{#1}{}}{~~}{~(#1)~~}%
	\mdseries\itshape\color{black}%
}%
{\par}
 %导入定理类环境重定义文件(需要amssymb、ifthen宏包)

%标题页设置
\title{
	有关矩阵和向量的导数
}
\author{王美庭\thanks{王美庭,暨南大学经济与社会研究院,Email: wangmeiting92@gmail.com}}
\date{\today}



%---------正文区-----------------
\begin{document}
\maketitle

\tableofcontents

\section{矩阵对实变量的导数}
设矩阵\footnote{在本文的符号习惯中,未加粗的字符代表标量;加粗的小写字符代表向量;加粗的大写字符代表矩阵。}
\[
\mathbf{A}(t) =
\begin{bmatrix}
	a_{11}(t) & a_{12}(t) & \cdots & a_{1n}(t) \\
	a_{21}(t) & a_{22}(t) & \cdots & a_{2n}(t) \\
	\vdots & \vdots & & \vdots \\
	a_{m1}(t) & a_{m2}(t) & \cdots & a_{mn}(t) \\
\end{bmatrix}
\]
定义矩阵对实变量的导数:
\[
\frac{\diff \mathbf{A}(t)}{\diff t} =
\begin{bmatrix}
	\frac{\diff a_{11}(t)}{\diff t} & \frac{\diff a_{12}(t)}{\diff t} & \cdots & \frac{\diff a_{1n}(t)}{\diff t} \\
	\frac{\diff a_{21}(t)}{\diff t} & \frac{\diff a_{22}(t)}{\diff t} & \cdots & \frac{\diff a_{2n}(t)}{\diff t} \\
	\vdots & \vdots & & \vdots \\
	\frac{\diff a_{m1}(t)}{\diff t} & \frac{\diff a_{m2}(t)}{\diff t} & \cdots & \frac{\diff a_{mn}(t)}{\diff t} \\
\end{bmatrix}
\]
则其有如下性质:
\begin{align}
\frac{\diff \left[ \mathbf{A}(t) \pm \mathbf{B}(t) \right]}{\diff t} &= \frac{\diff \mathbf{A}(t)}{\diff t} \pm \frac{\diff \mathbf{B}(t)}{\diff t} \\
\frac{\diff \left[ k(t) \mathbf{A}(t) \right]}{\diff t} &= \frac{\diff k(t)}{\diff t} \mathbf{A}(t) + k(t) \frac{\diff \mathbf{A}(t)}{\diff t} \\
\frac{\diff \left[ k(t) \mathbf{A}(t) \right]}{\diff t} &= \frac{\diff k(t)}{\diff t} \mathbf{A}(t) + k(t) \frac{\diff \mathbf{A}(t)}{\diff t} \\
\frac{\diff \left[ \mathbf{A}(t)\mathbf{B}(t) \right]}{\diff t} &= \frac{\mathbf{A}(t)}{\diff t} \mathbf{B}(t) + \mathbf{A}(t) \frac{\diff \mathbf{B}(t)}{\diff t} \\
\frac{\diff \mathbf{A}^{-1}(t)}{\diff t} &= - \mathbf{A}^{-1}(t) \frac{\mathbf{A}(t)}{\diff t} \mathbf{A}^{-1}(t) \\
\frac{\diff \mathbf{A}[t(x)]}{\diff x} &= \frac{\diff \mathbf{A}(t)}{\diff t} t^{\prime}(x) = t^{\prime}(x) \frac{\diff \mathbf{A}(t)}{\diff t} \\
\frac{\diff\,(\mathrm{tr}\mathbf{A})}{\diff t} &= \mathrm{tr}\left( \frac{\diff\mathbf{A}}{\diff t} \right)
\end{align}


\section{数量函数对于向量的导数}
设
\[
\mathbf{x} =
\begin{bmatrix}
x_1 \\ x_2 \\ \vdots \\ x_m
\end{bmatrix}\quad
\mathbf{y} =
\begin{bmatrix}
	y_1 \\ y_2 \\ \vdots \\ y_n
\end{bmatrix}\quad
\mathbf{A} =
\begin{bmatrix}
	a_{11} & a_{12} & \cdots & a_{1n} \\
	a_{21} & a_{22} & \cdots & a_{2n} \\
	\vdots & \vdots & & \vdots \\
	a_{m1} & a_{m2} & \cdots & a_{mn} \\
\end{bmatrix}
\]
定义数量函数 $ f $ 对向量的导数:
\[
\frac{\partial f}{\partial \mathbf{x}}
= \begin{bmatrix}
	\frac{\partial f}{\partial x_1} \\
	\frac{\partial f}{\partial x_2} \\
	\vdots \\
	\frac{\partial f}{\partial x_m}
\end{bmatrix}\quad
\frac{\partial f}{\partial \mathbf{x}^{\mathrm{T}}} = \begin{bmatrix}
	\frac{\partial f}{\partial x_1} &
	\frac{\partial f}{\partial x_2} &
	\cdots &
	\frac{\partial f}{\partial x_m}
\end{bmatrix}
\]
则其有如下性质:
\begin{align}
\frac{\partial \mathbf{a}^{\mathrm{T}} \mathbf{x}}{\partial \mathbf{x}} &= \mathbf{a} \\
\frac{\partial(\mathbf{u}(\mathbf{x}) \cdot \mathbf{v}(\mathbf{x}))}{\partial \mathbf{x}^{\mathrm{T}}} &= \mathbf{u}^{\mathrm{T}}(\mathbf{x}) \frac{\partial \mathbf{v}(\mathbf{x})}{\partial \mathbf{x}^{\mathrm{T}}}+\mathbf{v}^{\mathrm{T}}(\mathbf{x})\frac{\partial \mathbf{u}(\mathbf{x})}{\partial \mathbf{x}^{\mathrm{T}}} \\
\frac{\partial(\mathbf{u}(\mathbf{x}) \cdot \mathbf{A} \mathbf{v}(\mathbf{x}))}{\partial \mathbf{x}^{\mathrm{T}}} &= \mathbf{u}^{\mathrm{T}}(\mathbf{x}) \mathbf{A} \frac{\partial \mathbf{v}(\mathbf{x})}{\partial \mathbf{x}^{\mathrm{T}}}+\mathbf{v}^{\mathrm{T}}(\mathbf{x}) \mathbf{A}^{\mathrm{T}} \frac{\partial \mathbf{u}(\mathbf{x})}{\partial \mathbf{x}^{\mathrm{T}}} \\
\frac{\partial \mathbf{x}^{\mathrm{T}} \mathbf{x}}{\partial \mathbf{x}} &= 2 \mathbf{x} \\
\frac{\partial \mathbf{x}^{\mathrm{T}} \mathbf{A} \mathbf{x}}{\partial \mathbf{x}} &= \left(\mathbf{A}^{\mathrm{T}}+\mathbf{A}\right) \mathbf{x} \\
\frac{\partial \mathbf{a}^{\mathrm{T}} \mathbf{x} \mathbf{x}^{\mathrm{T}} \mathbf{b}}{\partial \mathbf{x}} &= \left(\mathbf{a} \mathbf{b}^{\mathrm{T}}+\mathbf{b} \mathbf{a}^{\mathrm{T}}\right) \mathbf{x} \\
\frac{\partial f\left[\mathbf{y}(\mathbf{x})\right]}{\partial \mathbf{x}} &= \frac{\partial \mathbf{y}^{\mathrm{T}}(\mathbf{x}) }{\partial \mathbf{x}} \frac{\partial f(\mathbf{y})}{\partial \mathbf{y}} \\
\frac{\partial\mathbf{y}^{\mathrm{T}}(\mathbf{x}) \mathbf{A} \mathbf{y}(\mathbf{x})}{\partial \mathbf{x}}&=\frac{\partial\mathbf{y}^{\mathrm{T}}(\mathbf{x})}{\partial \mathbf{x}}\left(\mathbf{A}+\mathbf{A}^{\mathrm{T}}\right) \mathbf{y}(\mathbf{x}) \\
\frac{\partial\mathbf{y}^{\mathrm{T}}(\mathbf{x}) \mathbf{A} \mathbf{z}(\mathbf{x})}{\partial \mathbf{x}}&=\frac{\partial \mathbf{y}^{\mathrm{T}}(\mathbf{x}) }{\partial \mathbf{x}} \mathbf{A} \mathbf{z}(\mathbf{x})+\frac{\partial \mathbf{z}^{\mathrm{T}}(\mathbf{x}) }{\partial \mathbf{x}} \mathbf{A}^{\mathrm{T}} \mathbf{y}(\mathbf{x}) \\
\frac{\partial(\mathbf{a}-\mathbf{A} \mathbf{x})^{\mathrm{T}} \mathbf{B}(\mathbf{a}-\mathbf{A} \mathbf{x})}{\partial \mathbf{x}}&=-2 \mathbf{A}^{\mathrm{T}} \mathbf{B}(\mathbf{a}-\mathbf{A} \mathbf{x}) \\
\frac{\partial \left[ c_{1} f(\mathbf{x})+c_{2} g(\mathbf{x}) \right] }{\partial \mathbf{x}} &= c_{1} \frac{\partial f(\mathbf{x})}{\partial \mathbf{x}}+c_{2} \frac{\partial g(\mathbf{x})}{\partial \mathbf{x}} \\
\frac{\partial f(\mathbf{x}) g(\mathbf{x})}{\partial \mathbf{x}} &= f(\mathbf{x}) \frac{\partial g(\mathbf{x})}{\partial \mathbf{x}}+\frac{\partial f(\mathbf{x})}{\partial \mathbf{x}} g(\mathbf{x}) \\
\frac{\partial \left[ f(\mathbf{x}) / g(\mathbf{x}) \right] }{\partial \mathbf{x}} &= \frac{1}{g^{2}(\mathbf{x})}\left[g(\mathbf{x}) \frac{\partial f(\mathbf{x})}{\partial \mathbf{x}}-f(\mathbf{x}) \frac{\partial g(\mathbf{x})}{\partial \mathbf{x}}\right] \\
\frac{\partial g(u(\mathbf{x}))}{\partial \mathbf{x}} &= \frac{\partial g(u)}{\partial u} \frac{\partial u(\mathbf{x})}{\partial \mathbf{x}} \\
\frac{\partial f(g(u(\mathbf{x})))}{\partial \mathbf{x}} &= \frac{\partial f(g)}{\partial g} \frac{\partial g(u)}{\partial u} \frac{\partial u(\mathbf{x})}{\partial \mathbf{x}}
\end{align}


\section{数量函数对于矩阵的导数}
设
\[
\mathbf{x} =
\begin{bmatrix}
	x_1 \\ x_2 \\ \vdots \\ x_m
\end{bmatrix}\quad
\mathbf{y} =
\begin{bmatrix}
	y_1 \\ y_2 \\ \vdots \\ y_n
\end{bmatrix}\quad
\mathbf{A} =
\begin{bmatrix}
	a_{11} & a_{12} & \cdots & a_{1n} \\
	a_{21} & a_{22} & \cdots & a_{2n} \\
	\vdots & \vdots & & \vdots \\
	a_{m1} & a_{m2} & \cdots & a_{mn} \\
\end{bmatrix}
\]
定义数量函数 $ f $ 对矩阵的导数:
\[
\frac{\partial f}{\partial \mathbf{A}}
= \begin{bmatrix}
	\frac{\partial f}{\partial a_{11}} & \frac{\partial f}{\partial a_{12}} & \cdots & \frac{\partial f}{\partial a_{1n}} \\
	\frac{\partial f}{\partial a_{21}} & \frac{\partial f}{\partial a_{22}} & \cdots & \frac{\partial f}{\partial a_{2n}} \\
	\vdots & \vdots & & \vdots \\
	\frac{\partial f}{\partial a_{m1}} & \frac{\partial f}{\partial a_{m2}} & \cdots & \frac{\partial f}{\partial a_{mn}} \\
\end{bmatrix}
\]
则其有如下性质:
\begin{align}
\frac{\partial \mathbf{x}^{\mathrm{T}} \mathbf{A} \mathbf{y}}{\partial \mathbf{A}} &= \mathbf{x y}^{\mathrm{T}} \\
\frac{\partial \mathbf{y}^{\mathrm{T}} \mathbf{A}^{\mathrm{T}} \mathbf{A} \mathbf{y}}{\partial \mathbf{A}} &= 2 \mathbf{A} \mathbf{y} \mathbf{y}^{\mathrm{T}} \\
\frac{\partial \mathbf{x}^{\mathrm{T}} \mathbf{A}^{-1} \mathbf{y}}{\partial \mathbf{A}}&=-\mathbf{A}^{-\mathrm{T}} \mathbf{x} \mathbf{y}^{\mathrm{T}} \mathbf{A}^{-\mathrm{T}} \\
\frac{\partial \mathbf{x}^{\mathrm{T}} \mathbf{A}^{\mathrm{T}} \mathbf{A} \mathbf{y}}{\partial \mathbf{A}}&=\mathbf{A}\left(\mathbf{x} \mathbf{y}^{\mathrm{T}}+\mathbf{y} \mathbf{x}^{\mathrm{T}}\right) \\
\frac{\partial \mathbf{x}^{\mathrm{T}} \mathbf{A} \mathbf{A}^{\mathrm{T}} \mathbf{y}}{\partial \mathbf{A}}&=\left(\mathbf{x} \mathbf{y}^{\mathrm{T}}+\mathbf{y} \mathbf{x}^{\mathrm{T}}\right) \mathbf{A} \\
z&=f(\mathbf{Y}), \mathbf{Y}=\mathbf{A} \mathbf{X}+\mathbf{B} \implies \frac{\partial z}{\partial \mathbf{X}}=\mathbf{A}^{\mathrm{T}} \frac{\partial z}{\partial \mathbf{Y}} \\
z&=f(\mathbf{Y}), \mathbf{Y}=\mathbf{X} \mathbf{A}+\mathbf{B} \Longrightarrow \frac{\partial z}{\partial \mathbf{X}}=\frac{\partial z}{\partial \mathbf{Y}} \mathbf{A}^{\mathrm{T}} \\
\frac{\partial\left[c_{1} f(\mathbf{A})+c_{2} g(\mathbf{A})\right]}{\partial \mathbf{A}}&=c_{1} \frac{\partial f(\mathbf{A})}{\partial \mathbf{A}}+c_{2} \frac{\partial g(\mathbf{A})}{\partial \mathbf{A}} \\
\frac{\partial f(\mathbf{A}) g(\mathbf{A})}{\partial \mathbf{A}}&=g(\mathbf{A}) \frac{\partial f(\mathbf{A})}{\partial \mathbf{A}}+f(\mathbf{A}) \frac{\partial g(\mathbf{A})}{\partial \mathbf{A}} \\
\frac{\partial \left[ f(\mathbf{A}) / g(\mathbf{A}) \right] }{\partial \mathbf{A}}&=\frac{1}{g^{2}(\mathbf{A})}\left[g(\mathbf{A}) \frac{\partial f(\mathbf{A})}{\partial \mathbf{A}}-f(\mathbf{A}) \frac{\partial g(\mathbf{A})}{\partial \mathbf{A}}\right] \\
\frac{\partial f(g(\mathbf{A}))}{\partial \mathbf{A}}&=\frac{\partial f}{\partial g} \frac{\partial g(\mathbf{A})}{\partial \mathbf{A}}
\end{align}

\subsection{矩阵的迹对于矩阵的导数}
矩阵的迹对于矩阵的导数有如下性质:
\begin{align}
\frac{\partial \operatorname{tr}(\mathbf{W})}{\partial \mathbf{W}}&=\mathbf{I} \\
\frac{\partial \operatorname{tr}\left(\mathbf{W}^{-1}\right)}{\partial \mathbf{W}} &= - \mathbf{W}^{-\mathrm{T}} \\
\frac{\partial \operatorname{tr}(\mathbf{W} \mathbf{A})}{\partial \mathbf{W}}&=\frac{\partial \operatorname{tr}(\mathbf{A} \mathbf{W})}{\partial \mathbf{W}}=\mathbf{A}^{\mathrm{T}} \\
\frac{\partial \operatorname{tr}\left(\mathbf{W}^{\mathrm{T}} \mathbf{A}\right)}{\partial \mathbf{W}}&=\frac{\partial \operatorname{tr}\left(\mathbf{A} \mathbf{W}^{\mathrm{T}}\right)}{\partial \mathbf{W}}=\mathbf{A} \\
\frac{\partial \operatorname{tr}\left(\mathbf{W} \mathbf{W}^{\mathrm{T}}\right)}{\partial \mathbf{W}}&=\frac{\partial \operatorname{tr}\left(\mathbf{W}^{\mathrm{T}} \mathbf{W}\right)}{\partial \mathbf{W}}=2 \mathbf{W} \\
\frac{\partial \operatorname{tr}\left(\mathbf{W}^{2}\right)}{\partial \mathbf{W}}&=\frac{\partial \operatorname{tr}(\mathbf{W} \mathbf{W})}{\partial \mathbf{W}}=2 \mathbf{W}^{\mathrm{T}} \\
\frac{\operatorname{tr}\left(\mathbf{A} \mathbf{W}^{-1}\right)}{\partial \mathbf{W}}&=-\left(\mathbf{W}^{-1} \mathbf{A} \mathbf{W}^{-1}\right)^{\mathrm{T}} \\
\frac{\partial \operatorname{tr}\left(\mathbf{W}^{\mathrm{T}} \mathbf{A} \mathbf{W}\right)}{\partial \mathbf{W}}&=\left(\mathbf{A}+\mathbf{A}^{\mathrm{T}}\right) \mathbf{W} \\
\frac{\partial \operatorname{tr}\left(\mathbf{W} \mathbf{A} \mathbf{W}^{\mathrm{T}}\right)}{\partial \mathbf{W}}&=\mathbf{W}\left(\mathbf{A}+\mathbf{A}^{T}\right) \\
\frac{\partial \operatorname{tr}\left(\mathbf{A} \mathbf{W}^{-1} \mathbf{B}\right)}{\partial \mathbf{W}}&=-\left(\mathbf{W}^{-1} \mathbf{B} \mathbf{A} \mathbf{W}^{-1}\right)^{\mathrm{T}} \\
\frac{\partial \operatorname{tr}\left(\mathbf{A} \mathbf{W} \mathbf{W}^{\mathrm{T}} \mathbf{A}^{\mathrm{T}}\right)}{\partial \mathbf{W}}&=2 \mathbf{A}^{\mathrm{T}} \mathbf{A} \mathbf{W} \\
\frac{\partial \operatorname{tr}\left(\mathbf{A} \mathbf{W}^{\mathrm{T}} \mathbf{W} \mathbf{A}^{\mathrm{T}}\right)}{\partial \mathbf{W}}&=2 \mathbf{W} \mathbf{A}^{\mathrm{T}} \mathbf{A} \\
\frac{\partial \operatorname{tr}\left(\mathbf{A} \mathbf{W} \mathbf{W}^{\mathrm{T}} \mathbf{B}\right)}{\partial \mathbf{W}}&=\left(\mathbf{B} \mathbf{A}+\mathbf{A}^{\mathrm{T}} \mathbf{B}^{\mathrm{T}}\right) \mathbf{W} \\
\frac{\partial \operatorname{tr}\left(\mathbf{A} \mathbf{W}^{\mathrm{T}} \mathbf{W} \mathbf{B}\right)}{\partial \mathbf{W}}&=\mathbf{W}\left(\mathbf{B} \mathbf{A}+\mathbf{A}^{\mathrm{T}} \mathbf{B}^{\mathrm{T}}\right) \\
\frac{\partial \operatorname{tr}\left(\mathbf{W A}^{\mathrm{T}} \mathbf{W}^{\mathrm{T}} \mathbf{B}\right)}{\partial \mathbf{W}}&=\mathbf{B} \mathbf{W} \mathbf{A}+\mathbf{B}^{\mathrm{T}} \mathbf{W} \mathbf{A}^{\mathrm{T}} \\
\frac{\partial \operatorname{tr}(\mathbf{W} \mathbf{A} \mathbf{W} \mathbf{B})}{\partial \mathbf{W}}&=\mathbf{B}^{\mathrm{T}} \mathbf{W}^{\mathrm{T}} \mathbf{A}^{\mathrm{T}}+\mathbf{A}^{\mathrm{T}} \mathbf{W}^{\mathrm{T}} \mathbf{B}^{\mathrm{T}}
\end{align}

\subsection{矩阵的行列式对于矩阵的导数}
矩阵的行列式对于矩阵的导数有如下性质:
\begin{align}
\frac{\partial|\mathbf{W}|}{\partial \mathbf{W}} &= |\mathbf{W}| \mathbf{W}^{-\mathrm{T}} \\
\frac{\partial\left|\mathbf{W}^{-1}\right|}{\partial \mathbf{W}}&=-|\mathbf{W}|^{-1}\mathbf{W}^{-\mathrm{T}} \\
\frac{\partial \ln |\mathbf{W}|}{\partial \mathbf{W}} &=\frac{1}{|\mathbf{W}|} \frac{\partial|\mathbf{W}|}{\partial \mathbf{W}} = \mathbf{W}^{-\mathrm{T}} \\
\frac{\partial\left|\mathbf{W}^{k}\right|}{\partial \mathbf{W}} &= k\left|\mathbf{W}^{k}\right| \mathbf{W}^{-\mathrm{T}} \\
\frac{\partial\left|\mathbf{W} \mathbf{W}^{\mathrm{T}}\right|}{\partial \mathbf{W}}&=2\left|\mathbf{W} \mathbf{W}^{\mathrm{T}}\right|\left(\mathbf{W} \mathbf{W}^{\mathrm{T}}\right)^{-1} \mathbf{W} \\
\frac{\partial\left|\mathbf{W}^{\mathrm{T}} \mathbf{W}\right|}{\partial \mathbf{W}}&=2\left|\mathbf{W}^{\mathrm{T}} \mathbf{W}\right| \mathbf{W}\left(\mathbf{W}^{\mathrm{T}} \mathbf{W}\right)^{-1} \\
\frac{\partial\left|\mathbf{W}^{2}\right|}{\partial \mathbf{W}}&=\mathbf{2}|\mathbf{W}|^{2} \mathbf{W}^{-\mathrm{T}} \\
\frac{\partial|\mathbf{A} \mathbf{W} \mathbf{B}|}{\partial \mathbf{W}}&=|\mathbf{A} \mathbf{W} \mathbf{B}| \mathbf{A}^{\mathrm{T}}\left(\mathbf{B}^{\mathrm{T}} \mathbf{W}^{\mathrm{T}} \mathbf{A}^{\mathrm{T}}\right)^{-1} \mathbf{B}^{\mathrm{T}} \\
\frac{\partial\left|\mathbf{W}^{\mathrm{T}} \mathbf{A} \mathbf{W}\right|}{\partial \mathbf{W}}&=2 \mathbf{A} \mathbf{W}\left(\mathbf{W}^{\mathrm{T}} \mathbf{A} \mathbf{W}\right)^{-1} \\
\frac{\partial\left|\mathbf{W} \mathbf{A} \mathbf{W}^{\mathrm{T}}\right|}{\partial \mathbf{W}}&=\left(\mathbf{W} \mathbf{A} \mathbf{W}^{\mathrm{T}}\right)^{-\mathrm{T}} \mathbf{W}\left(\mathbf{A}^{\mathrm{T}}+\mathbf{A}\right)
\end{align}


\section{向量对于向量的导数}
设
\[
\mathbf{x} =
\begin{bmatrix}
	x_1 \\ x_2 \\ \vdots \\ x_m
\end{bmatrix}\quad
\mathbf{y} =
\begin{bmatrix}
	y_1 \\ y_2 \\ \vdots \\ y_n
\end{bmatrix}
\]
定义向量对向量的导数:
\[
\frac{\partial\mathbf{y}}{\partial\mathbf{x}}
= \begin{bmatrix}
	\frac{\partial \mathbf{y}}{\partial x_1} \\
	\frac{\partial \mathbf{y}}{\partial x_2} \\
	\vdots \\
	\frac{\partial \mathbf{y}}{\partial x_m}
\end{bmatrix}\quad
\frac{\partial\mathbf{y}}{\partial\mathbf{x}^{\mathrm{T}}}
= \begin{bmatrix}
	\frac{\partial \mathbf{y}}{\partial x_1} &
	\frac{\partial \mathbf{y}}{\partial x_2} &
	\cdots &
	\frac{\partial \mathbf{y}}{\partial x_m}
\end{bmatrix}
\]
则其有如下性质:
\begin{align}
\frac{\partial \mathbf{A} \mathbf{x}}{\partial \mathbf{x}^{\mathrm{T}}} &= \mathbf{A} \\
\left[\frac{\partial \mathbf{y}(\mathbf{x})}{\partial \mathbf{x}^{\mathrm{T}}}\right]^{\mathrm{T}} &= \frac{\partial \mathbf{y}^{\mathrm{T}}(\mathbf{x})}{\partial \mathbf{x}} \\
\frac{\partial \mathbf{x}^{\mathrm{T}} \mathbf{A}}{\partial \mathbf{x}} &= \mathbf{A} \\
\frac{\partial \mathbf{x}^{\mathrm{T}}}{\partial \mathbf{x}} &= \frac{\partial \mathbf{x}}{\partial \mathbf{x}^{\mathrm{T}}} = \mathbf{I} \\
\frac{\partial v(\mathbf{x}) \mathbf{u}(\mathbf{x})}{\partial \mathbf{x}^{\mathrm{T}}} &= v(\mathbf{x}) \frac{\partial \mathbf{u}(\mathbf{x})}{\partial \mathbf{x}^{\mathrm{T}}}+\mathbf{u}(\mathbf{x}) \frac{\partial v(\mathbf{x})}{\partial \mathbf{x}^{\mathrm{T}}} \\
\frac{ \partial \left[ \mathbf{u}(\mathbf{x})+\mathbf{v}(\mathbf{x}) \right] }{\partial \mathbf{x}} &= \frac{\partial \mathbf{u}(\mathbf{x})}{\partial \mathbf{x}}+\frac{\partial \mathbf{v}(\mathbf{x})}{\partial \mathbf{x}} \\
\frac{\partial \mathbf{g}(\mathbf{u}(\mathbf{x}))}{\partial \mathbf{x}^{\mathrm{T}}} &= \frac{\partial \mathbf{g}(\mathbf{u})}{\partial \mathbf{u}^{\mathrm{T}}} \frac{\partial \mathbf{u}(\mathbf{x})}{\partial \mathbf{x}^{\mathrm{T}}} \\
\frac{\partial \mathbf{g}(\mathbf{u}(\mathbf{x}), \mathbf{v}(\mathbf{x}))}{\partial \mathbf{x}^{\mathrm{T}}}&=\frac{\partial \mathbf{g}(\mathbf{u}, \mathbf{v})}{\partial \mathbf{u}^{\mathrm{T}}} \frac{\partial \mathbf{u}(\mathbf{x})}{\partial \mathbf{x}^{\mathrm{T}}}+\frac{\partial \mathbf{g}(\mathbf{u}, \mathbf{v})}{\partial \mathbf{v}^{\mathrm{T}}} \frac{\partial \mathbf{v}(\mathbf{x})}{\partial \mathbf{x}^{\mathrm{T}}} \\
\frac{\partial \bm{f}(\mathbf{g}(\mathbf{u}(\mathbf{x})))}{\partial \mathbf{x}^{\mathrm{T}}}&=\frac{\partial \bm{f}(\mathbf{g})}{\partial \mathbf{g}^{\mathrm{T}}} \frac{\partial \mathbf{g}(\mathbf{u})}{\partial \mathbf{u}^{\mathrm{T}}} \frac{\partial \mathbf{u}(\mathbf{x})}{\partial \mathbf{x}^{\mathrm{T}}} \\
\frac{\partial \mathbf{A} \mathbf{u}(\mathbf{x})}{\partial \mathbf{x}^{\mathrm{T}}}&=\frac{\partial \mathbf{A} \mathbf{u}}{\partial \mathbf{u}^{\mathrm{T}}} \frac{\partial \mathbf{u}(\mathbf{x})}{\partial \mathbf{x}^{\mathrm{T}}}=\mathbf{A} \frac{\partial \mathbf{u}(\mathbf{x})}{\partial \mathbf{x}^{\mathrm{T}}}
\end{align}


\section{向量对于矩阵的导数}
设
\[
\mathbf{x} =
\begin{bmatrix}
	x_1 \\ x_2 \\ \vdots \\ x_m
\end{bmatrix}\quad
\mathbf{y} =
\begin{bmatrix}
	y_1 \\ y_2 \\ \vdots \\ y_n
\end{bmatrix}\quad
\mathbf{A} =
\begin{bmatrix}
	a_{11} & a_{12} & \cdots & a_{1n} \\
	a_{21} & a_{22} & \cdots & a_{2n} \\
	\vdots & \vdots & & \vdots \\
	a_{m1} & a_{m2} & \cdots & a_{mn} \\
\end{bmatrix}
\]
定义向量对于矩阵的导数:
\[
\frac{\partial\mathbf{x}}{\partial\mathbf{A}}
= \begin{bmatrix}
	\frac{\partial \mathbf{x}}{\partial a_{11}} & \frac{\partial \mathbf{x}}{\partial a_{12}} & \cdots & \frac{\partial \mathbf{x}}{\partial a_{1n}} \\
	\frac{\partial \mathbf{x}}{\partial a_{21}} & \frac{\partial \mathbf{x}}{\partial a_{22}} & \cdots & \frac{\partial \mathbf{x}}{\partial a_{2n}} \\
	\vdots & \vdots & \ddots & \vdots \\
	\frac{\partial \mathbf{x}}{\partial a_{m1}} & \frac{\partial \mathbf{x}}{\partial a_{m2}} & \cdots & \frac{\partial \mathbf{x}}{\partial a_{mn}}
\end{bmatrix}
\]
则其有如下性质:
\begin{align}
\frac{\partial \mathbf{A} \mathbf{y}}{\partial \mathbf{A}^{\mathrm{T}}}&=\mathbf{y} \otimes \mathbf{I} \\
\frac{\partial \mathbf{x}^{\mathrm{T}} \mathbf{A}}{\partial \mathbf{A}^{\mathrm{T}}} &= \mathbf{x}^{\mathrm{T}} \otimes \mathbf{I}
\end{align}


\section{矩阵对于矩阵的导数}
设
\[
\mathbf{A} =
\begin{bmatrix}
	a_{11} & a_{12} & \cdots & a_{1n} \\
	a_{21} & a_{22} & \cdots & a_{2n} \\
	\vdots & \vdots & & \vdots \\
	a_{m1} & a_{m2} & \cdots & a_{mn} \\
\end{bmatrix}\quad
\mathbf{B} =
\begin{bmatrix}
	b_{11} & b_{12} & \cdots & b_{1n} \\
	b_{21} & b_{22} & \cdots & b_{2n} \\
	\vdots & \vdots & & \vdots \\
	b_{m1} & b_{m2} & \cdots & b_{mn} \\
\end{bmatrix}
\]
定义矩阵对于矩阵的导数如下:
\[
\frac{\partial\mathbf{A}}{\partial\mathbf{B}} =
\begin{bmatrix}
	\frac{\partial \mathbf{A}}{\partial b_{11}} & \frac{\partial \mathbf{A}}{\partial b_{12}} & \cdots & \frac{\partial \mathbf{A}}{\partial b_{1n}} \\
	\frac{\partial \mathbf{A}}{\partial b_{21}} & \frac{\partial \mathbf{A}}{\partial b_{22}} & \cdots & \frac{\partial \mathbf{A}}{\partial b_{2n}} \\
	\vdots & \vdots & \ddots & \vdots \\
	\frac{\partial \mathbf{A}}{\partial b_{m1}} & \frac{\partial \mathbf{A}}{\partial b_{m2}} & \cdots & \frac{\partial \mathbf{A}}{\partial b_{mn}}
\end{bmatrix}
\]
则其有如下性质:
\begin{align}
\frac{\partial \left[ \mathbf{A}(\mathbf{X})+\mathbf{B}(\mathbf{X}) \right] }{\partial \mathbf{X}}&=\frac{\partial \mathbf{A}(\mathbf{X})}{\partial \mathbf{X}}+\frac{\partial \mathbf{B}(\mathbf{X})}{\partial \mathbf{X}}
\end{align}


\section{Hessian 矩阵}
设
\[
\mathbf{x} =
\begin{bmatrix}
	x_1 \\ x_2 \\ \vdots \\ x_m
\end{bmatrix}
\]
定义 Hessian 矩阵:
\[
\frac{\partial^{2} f(\mathbf{x})}{\partial \mathbf{x} \partial \mathbf{x}^{\mathrm{T}}}=\frac{\partial}{\partial \mathbf{x}^{\mathrm{T}}}\left[\frac{\partial f(\mathbf{x})}{\partial \mathbf{x}}\right] =
\begin{bmatrix}
	\frac{\partial^{2} f}{\partial x_{1}^{2}} & \frac{\partial^{2} f}{\partial x_{1} \partial x_{2}} & \cdots & \frac{\partial^{2} f}{\partial x_{1} \partial x_{m}} \\
	\frac{\partial^{2} f}{\partial x_{2} \partial x_{1}} & \frac{\partial^{2} f}{\partial x_{2}^{2}} & \cdots & \frac{\partial^{2} f}{\partial x_{2} \partial x_{m}} \\
	\vdots & \vdots & & \vdots \\
	\frac{\partial^{2} f}{\partial x_{m} \partial x_{1}} & \frac{\partial^{2} f}{\partial x_{m} \partial x_{2}} & \cdots & \frac{\partial^{2} f}{\partial x_{m}^{2}}
\end{bmatrix}
\]
则其有如下性质:
\begin{alignat}{2}
\frac{\partial^{2} \mathbf{a}^{\mathrm{T}} \mathbf{x}}{\partial \mathbf{x} \partial \mathbf{x}^{\mathrm{T}}}&=&&\mathbf{0} \\
\frac{\partial^{2} \mathbf{x}^{\mathrm{T}} \mathbf{A} \mathbf{x}}{\partial \mathbf{x} \partial \mathbf{x}^{\mathrm{T}}} &=&&\mathbf{A}^{\mathrm{T}}+\mathbf{A} \\
\frac{\partial^{2}(\mathbf{a}-\mathbf{A} \mathbf{x})^{\mathrm{T}} \mathbf{B}(\mathbf{a}-\mathbf{A} \mathbf{x})}{\partial \mathbf{x} \partial \mathbf{x}^{\mathrm{T}}}&=&&2 \mathbf{A}^{\mathrm{T}} \mathbf{B} \mathbf{A} \\
\frac{\partial^{2}\mathbf{y}^{\mathrm{T}}(\mathbf{x}) \mathbf{A} \mathbf{y}(\mathbf{x})}{\partial \mathbf{x} \partial \mathbf{x}^{\mathrm{T}}} &=&&\frac{\partial\mathbf{y}^{\mathrm{T}}(\mathbf{x})}{\partial \mathbf{x}}\left(\mathbf{A}+\mathbf{A}^{\mathrm{T}}\right) \frac{\partial \mathbf{y}(\mathbf{x})}{\partial \mathbf{x}^{\mathrm{T}}} + \notag \\
&&&\left[\mathbf{y}^{\mathrm{T}}(\mathbf{x})\left(\mathbf{A}+\mathbf{A}^{\mathrm{T}}\right) \otimes \mathbf{I}_{m}\right] \frac{\partial}{\partial \mathbf{x}^{\mathrm{T}}}\left\{\operatorname{vec}\left[\frac{\partial\mathbf{y}^{\mathrm{T}}(\mathbf{x})}{\partial \mathbf{x}}\right]\right\}
\end{alignat}


\section{拓展:矩阵的微分与其导数的关系}
\subsection{矩阵微分的常见性质}
\begin{align}
\diff\mathbf{A}&=\mathbf{0} \\
\diff\,(\alpha \mathbf{U})&=\alpha \diff \mathbf{U} \\
\diff\,(\mathbf{U} \pm \mathbf{V})&=\diff \mathbf{U} \pm \diff \mathbf{V} \\
\diff\,(\mathbf{UV})&=(\diff \mathbf{U}) \mathbf{V}+\mathbf{U}(\diff \mathbf{V}) \\
\diff\,(\mathbf{UVW})&=(\diff \mathbf{U}) \mathbf{VW}+\mathbf{U}(\diff \mathbf{V}) \mathbf{W} + \mathbf{UV}(\diff \mathbf{W}) \\
\diff\,(\mathbf{U} \otimes \mathbf{V})&=(\diff \mathbf{U}) \otimes \mathbf{V}+\mathbf{U} \otimes \diff \mathbf{V} \\
\diff\,(\mathbf{U} \odot \mathbf{V})&=(\diff \mathbf{U}) \odot \mathbf{V}+\mathbf{U} \odot \diff \mathbf{V} \\
\diff\,(\operatorname{vec}(\mathbf{U}))&=\operatorname{vec}(\diff \mathbf{U}) \\
\diff|\mathbf{X}|&=|\mathbf{X}| \operatorname{tr}\left(\mathbf{X}^{-1} \diff \mathbf{X}\right) \\
\diff\left(\mathbf{X}^{-1}\right)&=-\mathbf{X}^{-1}(\diff \mathbf{X}) \mathbf{X}^{-1}
\end{align}

\subsection{矩阵的数量函数与其微分的关系}

\begin{lemma}\label{lem:differential}%
若矩阵的数量函数 $ f(\mathbf{X}) $ 在 $ m\times n $ 矩阵点 $ \mathbf{X} $ 可微分,则下列关系成立:
\begin{gather*}
\diff f(\mathbf{X})=\mathbf{A} \diff \mathbf{X} \iff \nabla f(\mathbf{X})=\frac{\partial f(\mathbf{X})}{\partial \mathbf{X}}=\mathbf{A}^{\mathrm{T}} \\
\diff f(\mathbf{X})=\operatorname{tr}(\mathbf{A} \diff \mathbf{X}) \iff \nabla f(\mathbf{X})=\frac{\partial f(\mathbf{X})}{\partial \mathbf{X}}=\mathbf{A}^{\mathrm{T}}
\end{gather*}
式中, $\mathbf{A}$ 有可能与变元矩阵 $\mathbf{X}$ 有关。
\end{lemma}

由此我们可以推导前面数量函数对于向量或矩阵的导数的结论。下面给出几个例子:

\begin{exam}
用数量函数的微分形式,证明
\[
\frac{\partial \mathbf{x}^{\mathrm{T}} \mathbf{A} \mathbf{x}}{\partial \mathbf{x}}=\left(\mathbf{A}+\mathbf{A}^{\mathbf{T}}\right) \mathbf{x}
\]
\end{exam}

\begin{proof}
依据
\[
\diff\,(\mathbf{UVW})=(\diff \mathbf{U}) \mathbf{VW}+\mathbf{U}(\diff \mathbf{V}) \mathbf{W} + \mathbf{UV}(\diff \mathbf{W}) \\
\]
我们有
\begin{align*}
\diff\left(\mathbf{x}^{\mathrm{T}} \mathbf{A} \mathbf{x}\right)&=(\diff \mathbf{x})^{\mathrm{T}} \mathbf{A} \mathbf{x}+\mathbf{x}^{\mathrm{T}} \mathbf{A} \diff \mathbf{x} \\
&= \mathbf{x}^{\mathrm{T}} \mathbf{A}^{\mathrm{T}} \diff\mathbf{x}+\mathbf{x}^{\mathrm{T}} \mathbf{A} \diff \mathbf{x} \\
&= \mathbf{x}^{\mathrm{T}}\left(\mathbf{A}+\mathbf{A}^{\mathrm{T}}\right) \diff \mathbf{x}
\end{align*}
依据引理 \ref{lem:differential} 可得
\[
\nabla\left(\mathbf{x}^{\mathbf{T}} \mathbf{A} \mathbf{x}\right)=\frac{\partial \mathbf{x}^{\mathrm{T}} \mathbf{A} \mathbf{x}}{\partial \mathbf{x}} = \left[ \mathbf{x}^{\mathrm{T}}\left(\mathbf{A}+\mathbf{A}^{\mathrm{T}}\right) \right]^{\mathrm{T}} = \left(\mathbf{A}+\mathbf{A}^{\mathbf{T}}\right) \mathbf{x}
\]
证毕。
\end{proof}

\begin{exam}
用数量函数的微分形式,证明
\[
\frac{\partial \operatorname{tr}\left(\mathbf{X}^{\mathrm{T}} \mathbf{A} \mathbf{X}\right)}{\partial \mathbf{X}}=\left(\mathbf{A} + \mathbf{A}^{\mathrm{T}}\right) \mathbf{X}
\]
\end{exam}

\begin{proof}
\begin{align*}
	\diff\,\operatorname{tr}\left(\mathbf{X}^{\mathrm{T}} \mathbf{A} \mathbf{X}\right) &= \operatorname{tr}\left[ \diff\left( \mathbf{X}^{\mathrm{T}} \mathbf{A} \mathbf{X} \right) \right] \\
	&= \operatorname{tr}\left[ (\diff\mathbf{X})^{\mathrm{T}}\mathbf{AX} + \mathbf{X}^{\mathrm{T}} \mathbf{A} \diff\mathbf{X} \right] \\
	&= \operatorname{tr}\left[ (\diff\mathbf{X})^{\mathrm{T}}\mathbf{AX} \right] + \operatorname{tr}\left[ \mathbf{X}^{\mathrm{T}} \mathbf{A} \diff\mathbf{X} \right] \\
	&= \operatorname{tr}\left[ \mathbf{X}^{\mathrm{T}} \mathbf{A}^{\mathrm{T}} \diff\mathbf{X} \right] + \operatorname{tr}\left[ \mathbf{X}^{\mathrm{T}} \mathbf{A} \diff\mathbf{X} \right] \\
	&= \operatorname{tr}\left[ \mathbf{X}^{\mathrm{T}} \left( \mathbf{A} + \mathbf{A}^{\mathrm{T}} \right) \diff\mathbf{X} \right]
\end{align*}
依据引理 \ref{lem:differential} 可得
\[
\nabla\left[ \operatorname{tr}\left(\mathbf{X}^{\mathrm{T}} \mathbf{A} \mathbf{X}\right) \right] = \frac{\partial \operatorname{tr}\left(\mathbf{X}^{\mathrm{T}} \mathbf{A} \mathbf{X}\right)}{\partial \mathbf{X}} = \left[ \mathbf{X}^{\mathrm{T}} \left( \mathbf{A} + \mathbf{A}^{\mathrm{T}} \right) \right]^{\mathrm{T}} = \left( \mathbf{A} + \mathbf{A}^{\mathrm{T}} \right) \mathbf{X}
\]
证毕。
\end{proof}

\begin{exam}
用数量函数的微分形式,证明
\[
\frac{\partial\left|\mathbf{X}^{\mathrm{T}} \mathbf{X}\right|}{\partial \mathbf{X}}=2\left|\mathbf{X}^{\mathrm{T}} \mathbf{X}\right| \mathbf{X}\left(\mathbf{X}^{\mathrm{T}} \mathbf{X}\right)^{-1}
\]
\end{exam}

\begin{proof}
\begin{align*}
	\diff \left| \mathbf{X}^{\mathrm{T}}\mathbf{X} \right|
	&= \left| \mathbf{X}^{\mathrm{T}}\mathbf{X} \right| \operatorname{tr} \left[ \left( \mathbf{X}^{\mathrm{T}}\mathbf{X} \right)^{-1} \diff\left( \mathbf{X}^{\mathrm{T}}\mathbf{X} \right) \right] \\
	&= \left| \mathbf{X}^{\mathrm{T}}\mathbf{X} \right| \operatorname{tr} \left[ \left( \mathbf{X}^{\mathrm{T}}\mathbf{X} \right)^{-1} \left(\diff\mathbf{X}\right)^{\mathrm{T}} \mathbf{X} + \left( \mathbf{X}^{\mathrm{T}}\mathbf{X} \right)^{-1} \mathbf{X}^{\mathrm{T}} \diff\mathbf{X} \right] \\
	&= \operatorname{tr} \left[ 2\left| \mathbf{X}^{\mathrm{T}}\mathbf{X} \right| \left( \mathbf{X}^{\mathrm{T}}\mathbf{X} \right)^{-1} \mathbf{X}^{\mathrm{T}} \diff\mathbf{X} \right]
\end{align*}
依据引理 \ref{lem:differential} 可得
\[
\nabla \left| \mathbf{X}^{\mathrm{T}}\mathbf{X} \right|
= \frac{\partial\left|\mathbf{X}^{\mathrm{T}} \mathbf{X}\right|}{\partial \mathbf{X}}
= \left[ 2\left| \mathbf{X}^{\mathrm{T}}\mathbf{X} \right| \left( \mathbf{X}^{\mathrm{T}}\mathbf{X} \right)^{-1} \mathbf{X}^{\mathrm{T}} \right]^{\mathrm{T}}
=2\left|\mathbf{X}^{\mathrm{T}} \mathbf{X}\right| \mathbf{X}\left(\mathbf{X}^{\mathrm{T}} \mathbf{X}\right)^{-1}
\]
以上系列方程使用了以下性质:
\begin{gather*}
\operatorname{tr}(\mathbf{AB})=\operatorname{tr}(\mathbf{BA}) \\
\operatorname{tr}\left(\mathbf{A}^{\mathrm{T}} \mathbf{B}\right)=\operatorname{tr}\left(\mathbf{B}^{\mathrm{T}} \mathbf{A}\right) \\
\left[\left( \mathbf{X}^{\mathrm{T}}\mathbf{X} \right)^{-1}\right]^{\mathrm{T}}=\left[\left(\mathbf{X}^{\mathrm{T}}\mathbf{X}\right)^{\mathrm{T}}\right]^{-1}= \left(\mathbf{X}^{\mathrm{T}}\mathbf{X}\right)^{-\mathrm{T}} =\left(\mathbf{X}^{\mathrm{T}}\mathbf{X}\right)^{-1}
\end{gather*}
证毕。
\end{proof}


\section{参考资料}
\setlength{\parindent}{0em}
[1] 张贤达. 矩阵分析与应用[M]. 清华大学出版社有限公司, 2004.

\hangindent = 1.45em \hangafter = 1
[2] Petersen K B, Pedersen M S. The matrix cookbook (version: November 15, 2012)[J]. 2012.

[3] 印象笔记:矩阵向量求导haha





































\end{document}

